\documentclass[12pt, a4paper, notitlepage, oneside]{article}

\usepackage{caption}
\usepackage{dirtytalk}
\usepackage{siunitx}
\usepackage{graphicx}

\author{Aravind Vasudevan}
\date{{\small \today}}
\title{{\Huge The man who cracked the impossible}}

\begin{document}

%\pagestyle{headings}

\maketitle


%\vspace*{2em} 
\begin{quote} 
\centering 
\emph{\say{Sometimes it is the people no one can imagine anything of who do the things no one can imagine.}} \\ \textemdash \hspace{2mm} \textit{Alan Turing}
\end{quote}
\vspace*{2em}

Code Breaking played a fundamental part in the result of the Second World War. The Germans invented the \emph{Enigma Machine}, which was at the time, generated the most complicated ciphers. It was used to encrypt military information transmitted over the radio during the war. It had about \num{150738274937250} states and about \emph{159 quintillion} combinations and the key used in it was changed once in \emph{\num{24} hours}, making it the one of impenetrable code to decrypt. \\

But yet, there was one mind that did succeed. Alan Mathison Turing was an English computer scientist, who is widely considered to be the father of theoretical computer science and artificial intelligence. During the Second World War, Turing worked for the Government Code and Cypher School (GC\&CS) at Bletchley Park, Britain's codebreaking center. Building on Polish research into the Enigma code, he and mathematician Gordon Welchman developed an electromechanical machine called a \textit{Bombe}. Though the Poles had succeeded in reading Enigma messages on the simplest key systems, this machine allowed any message to be deciphered, so long as the hardware of the Enigma was known and a plain-text ‘crib’ of about 20 letters could be guessed correctly. It is believed that Turing’s work shortened the war in Europe by at least two years. \\

Alan Turing has also worked on other key inventions of Computer Science such as the \emph{Turing Machine} and the \emph{LU decomposition}. A Turing machine is an abstract machine that manipulates symbols on a strip of tape according to a table of rules; to be more exact, it is a mathematical model of computation that defines such a device. The machine operates on an infinite memory tape divided into discrete cells. The machine positions its head over a cell and reads the symbol there. Then as per the symbol and its present place in a finite table of user-specified instructions the machine writes a symbol in the cell, then either moves the tape one cell left or right and either proceeds to a subsequent instruction or halts the computation. Despite its simplicity, the machine can simulate any computer algorithm, no matter how complicated it is. \\

Turing, during his work at Bletchley Park, mostly preferred to work alone, due to his introvert nature. He had a hard time expressing himself mostly because of his Homosexuality, which was still a crime in the UK and he had to conceal it away. He was arrested and came to trial on 31 March 1952, after the police learned of his sexual relationship with a young Manchester man. He made no serious denial or defense, instead telling everyone that he saw no wrong with his actions. He accepted chemical castration treatment, with DES, as an alternative to prison. Turing committed suicide in 1954, aged 41. In 2009, British Prime Minister Gordon Brown issued an official apology for his prosecution, saying \say{you deserved so much better.}


\end{document}